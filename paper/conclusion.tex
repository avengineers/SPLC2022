\section{Conclusion and Future Work}\label{conclusion}

This work provides an iterative and incremental migration strategy towards SPL
for the software of a real-world product family of an automotive supplier. We
proposed three steps to perform the migration of a project's software repository
to an SPL variant: (1) \textit{Complexity Reduction}, (2)
\textit{Modularization} and (3) \textit{Versioned Dependency Management}. It was
shown that our proposal can be applied to the repositories of a product family's
software at any time of the development phase due to the fact that any of the
three steps can be applied incrementally and each transformed variant can
iteratively continue with the next migration step independent of the others.
With our proposal we:

\begin{itemize}
  \item established a stable SPL with variants buildable at any time,
  \item reduced workload of developers by reducing the number of repositories,
  \item increased collaboration beyond project borders and
  \item integrated all software sources of a product family into one git
        repository.
\end{itemize}

We implemented a build system with modern CMake fully supporting the concept of
SPL variants.

At the submission date's point of time we managed to proof about 50\% of our
concept. The build system prototype required two persons full-time for six
weeks, the implementation for migration step one and two required two persons
full-time for three more months. Over a time frame of two months we
incrementally added ten variants, while each import to the SPL was really fast
due to the automated importing mechanism. We already started with
\textit{Modularization} of software components, but only converted a few
components and begun with removing duplicates. Recently we focussed more on
automation than on the SPL migration, as we saw more advantages there in the
long run. The automation for CI/CD is helpful for the migration as well.
Nevertheless, the build system requires some more features to be able to handle
our migration step three and we plan to implement them in the coming months, in
parallel to the \textit{Modularization} step of the software components.

Additionally to the build system implementation and the actual migration of the
embedded source code, it is required for us to qualify tools which have an
impact on the created binaries or the processes to build them. This tool
qualification is requested by the ISO 26262 standard for safety reasons. We must
take care that CMake, Ninja and KConfig are properly qualified. On the other
hand, proprietary software might come with safety evaluations already. This
should be kept in mind, when deciding for your tool landscape.

Besides all the discussed technical challenges, there were also social obstacles
on the way to SPL. Fear was probably the strongest emotion we faced. At the
beginning we were certain that all required build system features were
implemented and we always had a backup plan to go back to the original
repositories. There was no real threat and still no one had enough confidence to
start the migration. Afterwards we do not see a reason for hesitation, but
sometimes it simply needs someone brave in the team or someone pushing you. Our
conclusion is: Just do it! And if you break it, make sure you can fix it fast!
