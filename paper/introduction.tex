\section{Introduction}\label{introduction}

Development of product families rather than individual products is a goal for
many engineering companies. Product Line Engineering (PLE) is a widely used
approach for reaching this goal. It is common and accepted for electronic and
mechanical developments and thus especially important for the automotive
industry that historically comes from this area~\cite{bookssp19X19}. However,
for quite some time the automotive industry is constantly moving its focus
towards the development of software. This development is still ongoing and
becomes more and more complex each year~\cite{ICSE03498}. In order to stay
competitive in the future and to realize a competitive advantage, a fast
adaption to the market and to customer needs is also required for software.
Offering off-the-shelf software products with tailored customer features will
increase the company's revenue. Fortunately PLE has also been applied to
software engineering for many years already in the form of Software Product
Lines (SPL)~\cite{confsplc2000}. An SPL increases reusability~\cite{Clem02a} and
therefore increases quality and decreases development costs of software by
building up shared core assets and customer-specific functionality. With this
reuse in place, a business will be able to do larger scaling, with higher number
of projects the development costs will drastically decrease compared to
traditional engineering methods and the time-to-market will improve due to less
development effort~\cite{confsplcAzanzaMD21}. Thus an SPL is one possible
solution to win against competitors.

But even after years of research, source code migrations to SPL are still
complicated and there are no bullet proof concepts available that are applicable
to actively running projects with tight release plans. Much research has been
done on product line architectures~\cite{Svahnberg1999EPLA,
confsplcTomashchukLJ21} and system engineering~\cite{confsplcSchaferBAKR21},
about feature model migrations~\cite{ncstrlustuttgartfiINPROC200185,
confsplcGrunerBKR20, confsplcDuszynskiDB19, confsplcFritschAR20}, SPL
evolution~\cite{journalssmrQuintonVRBGS21, Svahnberg1999ESPL, Eise02b,
kconfigKernel} and safety aspects~\cite{confsplcWolschke0SAM19} of SPLs. On the
other hand, \cite{confoopslaHetrickKM06} and \cite{confsplcAbbasJLESS20} have
written about the overall process, general benefits and a change of mindset, but
not specifically about source code. Contrary to \cite{confsplcRubinCC13}, cloned
variants are no option for us anymore. The authors of \cite{Krueger2001SMC}
reported different migration strategies more than 20 years ago already: (1)
proactive, (2) extractive and (3) reactive, but no work described a mixture of
extractive and reactive migrations, which we needed. Inspired by the work of
\cite{confsplcJepsenDB07} we aim for a strategy with which we are able to
migrate multiple independent repositories to a scalable SPL, allowing developers
and software product managers to proceed with their own pace and according to
the project plan without risking its deadlines and always being able to do a
step backwards if neccessary.

In this paper we propose an iterative and incremental migration strategy in
three steps. We structured the paper in the following way:
Section~\ref{challenge} stating the challenges we faced in our projects and
industry to explain our view on the problem and why standard solutions did not
work for us, section~\ref{solution} explaining our three step solution of the
source code migration in detail, presenting infrastructure and build system
ideas and section \ref{conclusion} presenting our conclusion, an overview on
where we stand right now and giving a forecast on future work.
