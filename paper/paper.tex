\documentclass[sigconf]{acmart}

\usepackage{booktabs}
\usepackage{balance}
\usepackage{graphicx,lipsum}
\usepackage{fancyvrb}
\graphicspath{ {./images/} }

\acmConference[SPLC'22]{26th International Systems and Software Product Line Conference}{12--16 September, 2022}{Graz, Austria}

\begin{document}

\title{In Three Steps to Software Product Lines: a Practical Example from the Automotive Industry}

\author{Matthias Eggert}
\affiliation{%
    \department{Rhine-Main-Team (RMT)}
    \institution{Marquardt GmbH}
    \streetaddress{Schloss-Str. 16}
    \postcode{78604}
    \city{Rietheim-Weilheim}
    \country{Germany}
}
\email{matthias.eggert@marquardt.com}
\author{Karsten Günther}
\affiliation{%
    \department{Rhine-Main-Team (RMT)}
    \institution{Marquardt GmbH}
    \streetaddress{Schloss-Str. 16}
    \postcode{78604}
    \city{Rietheim-Weilheim}
    \country{Germany}
}
\email{karsten.guenther@marquardt.com}
\author{Jochen Maletschek}
\affiliation{%
    \department{Rhine-Main-Team (RMT)}
    \institution{Marquardt GmbH}
    \streetaddress{Schloss-Str. 16}
    \postcode{78604}
    \city{Rietheim-Weilheim}
    \country{Germany}
}
\email{jochen.maletschek@marquardt.com}
\author{Alexandru Maxiniuc}
\affiliation{%
    \department{Rhine-Main-Team (RMT)}
    \institution{Marquardt GmbH}
    \streetaddress{Schloss-Str. 16}
    \postcode{78604}
    \city{Rietheim-Weilheim}
    \country{Germany}
}
\email{alexandru.maxiniuc@marquardt.com}
\author{Alexander Mann-Wahrenberg}
\affiliation{%
    \department{Rhine-Main-Team (RMT)}
    \institution{Marquardt GmbH}
    \streetaddress{Schloss-Str. 16}
    \postcode{78604}
    \city{Rietheim-Weilheim}
    \country{Germany}
}
\email{alexander.mann-wahrenberg@marquardt.com}

\newpage

\begin{abstract}
    In the automotive industry, suppliers aim to increase their revenue and try
    to keep up with the pace of the market trends to stay competitive by
    offering off-the-shelf products to car manufacturers. On the other hand
    those car manufacturers request tailored products to gain unique selling
    points. Each new customer request may result in a new software project. To
    save time one might find it a good idea to create the new software project
    as a copy of an older one. This method guarantees initial functionality, but
    prevents refactoring and leads to continuous software erosion. The
    implementations diverge from each other and improvements cannot be shared.
    Software Product Lines (SPL) can help to maximize reusability and quality by
    building up shared core assets and customer-specific functionality. In our
    paper, we propose a method to migrate a customer project landscape into a
    scalable SPL in three steps.
\end{abstract}

\begin{CCSXML}
 <ccs2012>
    <concept>
        <concept_id>10011007.10011074.10011092.10011096.10011097</concept_id>
        <concept_desc>Software and its engineering~Software product lines</concept_desc>
        <concept_significance>500</concept_significance>
    </concept>
    <concept>
        <concept_id>10011007.10011074.10011092.10011096</concept_id>
        <concept_desc>Software and its engineering~Reusability</concept_desc>
        <concept_significance>500</concept_significance>
    </concept>
    <concept>
        <concept_id>10011007.10011074.10011111.10011113</concept_id>
        <concept_desc>Software and its engineering~Software evolution</concept_desc>
        <concept_significance>300</concept_significance>
    </concept>
    <concept>
        <concept_id>10011007.10011074.10011081.10011082.10011087</concept_id>
        <concept_desc>Software and its engineering~V-model</concept_desc>
        <concept_significance>100</concept_significance>
    </concept>
</ccs2012>
\end{CCSXML}



\ccsdesc[500]{Software and its engineering~Software product lines}
\ccsdesc[500]{Software and its engineering~Reusability}
\ccsdesc[300]{Software and its engineering~Software evolution}
\ccsdesc[100]{Software and its engineering~V-model}

\keywords{Software Product Line, migration, adoption, incremental migration, automotive, functionality reuse, active projects}

\maketitle

%%%%%%%%%%%%%%%%%%%%%%%%%%%%%%%%%%%%%%%%%%%%%%%%%%%%%%%%%%%%%%%%%%
%%% place content here to separate real paper from setup stuff %%%
%%%%%%%%%%%%%%%%%%%%%%%%%%%%%%%%%%%%%%%%%%%%%%%%%%%%%%%%%%%%%%%%%%
\section{Introduction}\label{introduction}

Development of product families rather than individual products is a goal for
many engineering companies. Product Line Engineering (PLE) is a widely used
approach for reaching this goal. It is common and accepted for electronic and
mechanical developments and thus especially important for the automotive
industry that historically comes from this area~\cite{bookssp19X19}. However,
for quite some time the automotive industry is constantly moving its focus
towards the development of software. This development is still ongoing and
becomes more and more complex each year~\cite{ICSE03498}. In order to stay
competitive in the future and to realize a competitive advantage, a fast
adaption to the market and to customer needs is also required for software.
Offering off-the-shelf software products with tailored customer features will
increase the company's revenue. Fortunately PLE has also been applied to
software engineering for many years already in the form of Software Product
Lines (SPL)~\cite{confsplc2000}. An SPL increases reusability~\cite{Clem02a} and
therefore increases quality and decreases development costs of software by
building up shared core assets and customer-specific functionality. With this
reuse in place, a business will be able to do larger scaling, with higher number
of projects the development costs will drastically decrease compared to
traditional engineering methods and the time-to-market will improve due to less
development effort~\cite{confsplcAzanzaMD21}. Thus an SPL is one possible
solution to win against competitors.

But even after years of research, source code migrations to SPL are still
complicated and there are no bullet proof concepts available that are applicable
to actively running projects with tight release plans. Much research has been
done on product line architectures~\cite{Svahnberg1999EPLA,
confsplcTomashchukLJ21} and system engineering~\cite{confsplcSchaferBAKR21},
about feature model migrations~\cite{ncstrlustuttgartfiINPROC200185,
confsplcGrunerBKR20, confsplcDuszynskiDB19, confsplcFritschAR20}, SPL
evolution~\cite{journalssmrQuintonVRBGS21, Svahnberg1999ESPL, Eise02b,
kconfigKernel} and safety aspects~\cite{confsplcWolschke0SAM19} of SPLs. On the
other hand, \cite{confoopslaHetrickKM06} and \cite{confsplcAbbasJLESS20} have
written about the overall process, general benefits and a change of mindset, but
not specifically about source code. Contrary to \cite{confsplcRubinCC13}, cloned
variants are no option for us anymore. The authors of \cite{Krueger2001SMC}
reported different migration strategies more than 20 years ago already: (1)
proactive, (2) extractive and (3) reactive, but no work described a mixture of
extractive and reactive migrations, which we needed. Inspired by the work of
\cite{confsplcJepsenDB07} we aim for a strategy with which we are able to
migrate multiple independent repositories to a scalable SPL, allowing developers
and software product managers to proceed with their own pace and according to
the project plan without risking its deadlines and always being able to do a
step backwards if neccessary.

In this paper we propose an iterative and incremental migration strategy in
three steps. We structured the paper in the following way:
Section~\ref{challenge} stating the challenges we faced in our projects and
industry to explain our view on the problem and why standard solutions did not
work for us, section~\ref{solution} explaining our three step solution of the
source code migration in detail, presenting infrastructure and build system
ideas and section \ref{conclusion} presenting our conclusion, an overview on
where we stand right now and giving a forecast on future work.

\section{Challenge}\label{challenge}

Our migration method is supposed to be a generic solution, but our company's
circumstances might have some influence on the ideas that we share in this
paper. This section will explain the challenges we had, both coming from the
products' nature and its industry and also from our corporate processes and
strategies.

The projects that we migrated are in the area of electric cars for premium
automobile manufacturers. The software is written in (Embedded) C and Matlab and
modelled with AUTomotive Open System ARchitecture (AUTOSAR) with C code
generators. These projects must fulfill Automotive Software Process Improvement
and Capability Determination (ASPICE), ISO26262 and other processes, standards
and regulations of the automotive industry during their development phases to
stay competitive and meet legal requirements. A company will less likely get a
new inquiry when the standards and norms are not fulfilled. The standards are
relevant for the entire SPL development and do not only apply for the delivered
software product, but also for supporting processes and tools like the build
system. Additionally, the build system must be able to handle third party source
code deliveries without modifications. One example for this are AUTOSAR
deliveries. The dependency to a delivered AUTOSAR package is not only a
challenge for the toolchain, but also a chance. Due to AUTOSAR's layered
architecture and its interface and component design, it already provides modular
abstraction for separation of concerns, that can help in building SPL core
assets and to maximize functional reuse.

All migrated projects of our SPL base on the same product, a complex sensor
cluster. For historic reasons every (customer) project was managed as a separate
Dimensions repository. Dimensions is a Source Code Management (SCM) system
similar to Revision Control System (RCS) used at the Marquardt GmbH for source
code administration and project documentation. Each individual source code
repository is a self-contained ready-to-build project, including all required
tools and libraries to build the product binaries, like compilers, MSYS and GNU
Make. The C code was validated with Jenkins nightly builds. Those Jenkins
scripts were located in separate repositories, not part of the project's
repository itself. Unit tests were not running with every change of the source
code as part of CI/CD, but were running on demand.

Whenever a new customer project was kicked-off, it also triggered the start of a
new software project. This software project then started as a clone of another
already existing project (`clone-and-own'). This approach had several benefits:
the setup time was short and because a mature project was taken as basis, the
project's software was stable already and debugging and adapting was possible.
There was less work for generic parts and only hardware and customer-specific
modifications were required. However, this approach also caused a lot of
disadvantages. By simply cloning an existing project there was not only a copy
of functionality available in the new project, but also a copy of all its flaws.
Because of a large number of copies a possible refactoring of reused code had to
be merged back and this meant a high effort. If refactoring is never done, this
approach can result in software erosion, as it did in our projects. The software
erosion noticeably led to higher maintenance efforts at the end phase of the
development and brought up errors late during the product lifecycle. For all the
teams working in these projects, with the problems mentioned above, the idea of
migrating to another platform including a new tool chain, not only sounded like
a mammoth task but unrealistic too, having in mind the tightly planned release
schedules.

With our migration to SPL we try to tackle as many as possible of the mentioned
challenges. The migration is only partially done and still ongoing, but we did
not detect major issues so far. At the same time, we managed to:
\begin{itemize}
  \item migrate `clone-and-own' sources to our SPL,
  \item add variant handling capability to the build system,
  \item replace the outdated SCM system by Git and Bitbucket,
  \item introduce a CI/CD system,
  \item introduce a unit test framework as integral part of the development,
  \item automate the setup of the build environment and reduce the repository size drastically,
  \item stay compliant with all rules and norms,
  \item train the team and
  \item still keep release plans.
\end{itemize}

We will not write about all solution aspects in this paper. The focus will be
on the source code migration process and less on the tools, processes and
performed trainings. Our build system implementation is freely available on
Github.com~\cite{GithubSPL}.

\section{Solution}\label{solution}

Our goal with the migration to an SPL is to switch the mindset from `my project'
to `one of our product variants' by enabling the software teams to collaborate
in a single Git repository and be able to use all variants in a single IDE
instance to build any target of any product variant. Our SPL implementation uses
Visual Studio Code plus CMake Tools extension as IDE, CMake and Ninja for
building all required targets, Powershell and Python as build wrapper and CI/CD
test runners.

With the help of our SPL implementation we follow a migration strategy that
consists of three consecutive steps as depicted in Figure~\ref{fig:migrationFlow}.

\begin{figure}[htb]
  \centering
  \includegraphics[width=1\columnwidth]{migration-steps-overview.pdf}
  \caption{Migration Flow Overview}
  \Description{Migration Flow Overview showing the three proposed migration steps}
  \label{fig:migrationFlow}
\end{figure}

In an SPL we talk about at least two variation dimensions: space,
time~\cite{appliedSPLE} and sometimes also maturity~\cite{bigleverwhitepaper}.
Our first migration step \textit{Complexity Reduction} is merely a step to
prepare these dimensions, to bring code together that belongs together within a
new SPL-capable build system. The second step \textit{Modularization} enhances
the space dimension by building up configurable sources or variants of software
components to reduce lines of code and maximize reuse. With variation in space
we are able to create feature variants of our software and its components and
this is the beginning of the SPL idea. The third step \textit{Versioned
Dependency Management} focuses the time dimension and cross-product component
reuse.

All three steps will be explained in detail in the following subsections.
It is important to know though, that each step can be done incrementally:
\begin{itemize}
  \item \textit{Complexity Reduction}: variant by variant.
  \item \textit{Modularization}: component by component.
  \item \textit{Versioned Dependency Management}: component by component.
\end{itemize}

If one step is accomplished for a variant or a component, the migration of that
specific variant or component can iteratively proceed with the next migration
step.

The content of the SPL repository, especially the number of lines of code (LOC),
will change during the different migration steps as shown in
Figure~\ref{fig:threeMigrationSteps}. In the first migration step
\textit{Complexity Reduction} the LOC will increase with every legacy project
added to the codebase. During \textit{Modularization} legacy sources will be
exchanged with configurable sources leading to a reduction of duplicated code
and therefore to a decreased LOC.\@ Finally even the LOC of configurable sources
in the SPL repository will decrease, as the sources are isolated and moved to a
separate repository during the third step \textit{Versioned Dependency
Management}. Then those configurable sources will just be configured as external
build dependencies in several product families.

\begin{figure*}[ht]
  \centering
  \includegraphics[width=1.65\columnwidth]{migration_steps.pdf}
  \caption{Three Migration Steps to SPL}
  \Description{Three Migration Steps to SPL}
  \label{fig:threeMigrationSteps}
\end{figure*}

\subsection{Complexity Reduction}\label{complexity}

Our first step towards SPL consists of importing several active sensor projects
into the SPL, i.e., adding Dimension rdepository product variants into one
structured Git repository. To support further incremental imports of additional
sensor variants we implemented a tool called
Transformer~\cite{GithubTransformer} that highly automates the necessary import
steps. The Transformer is implemented in Python and can be applied to each
variant separately, therefore making the transformation incremental and
independent of already existing or upcoming variants. The transformation of a
legacy project into a new product variant is divided into four parts:

\begin{enumerate}
  \item \textbf{Copy sources}: The variant's original source tree is copied
        into a variant-specific legacy tree without any further changes or
        adaptions.
  \item \textbf{Create build configuration}: The original project build
        system (GNU Make) is used to extract all variant-specific configuration and
        to transform it into a CMake configuration and part list compatible with
        the SPL build system.
  \item \textbf{Create IDE configuration}: For seamless integration of the SPL
        repository and its variant concept into Visual Studio Code the
        CMake Tools extension is used. This extension introduces the concept of CMake
        Variants, therefore we create a configuration file with build settings
        for each variant.
  \item \textbf{Establish CI/CD}: In order to simplify the switch to the new
        tool environment, introduce a CI/CD system to increase automation and
        validation for every source code change.
\end{enumerate}

Switching from variant-specific repositories in Dimensions to a single Git
repository in Bitbucket causes a high workflow impact for the developers.
Additionally they need to get used to an SPL build system and a new IDE.\@
Therefore, we decided not to change the production source code during this first
migration step so that the developers can continue to work in their well-known
code basis.

The DevOps Handbook~\cite{devopshandbook} calls this a monolithic approach and
positively emphasizes the simplicity and resource efficiency in small scale. It
also mentions the drawbacks of overall poor scaling and weak modularization
capabilities. We are aware of those drawbacks and are going to tackle them
within the third migration step \textit{Versioned Dependency Management}.

The structure after the transformation looks like this:
\begin{Verbatim}[frame=single,samepage=true]
.vscode/
  cmake-kits.json
  cmake-variants.json
  settings.json
legacy/
  variant-a/
    component-1/
    component-2/
  variant-b/
    component-1/
variants/
  variant-a/
    config.cmake
    parts.cmake
  variant-b/
    config.cmake
    parts.cmake
CMakeLists.txt
\end{Verbatim}

The `legacy' folder contains all the projects' unmodified source code in
variant-specific folders. The configuration and part lists of the variants can
be found in the `variants' folder.

Although the build system and tool environment is reused already for all
variants, for source code there is no reuse in place and the LOC is growing over
time as shown in Figure~\ref{fig:threeMigrationSteps}, therefore scaling may
become problematic. Still there are a lot of advantages with the monolithic way,
like centralized deployment with our CI/CD system, start learning to use Git as
source code management system and reducing complexity by reducing the number of
systems: polyrepo vs.\ monorepo with a single build system and tool environment.

\subsection{Modularization}\label{modularization}

As soon as all developers, software project managers and integrators are used to
the outcome of the first step of the SPL migration, the \textit{Modularization}
step can begin. With the following three stages it is possible to switch from a
project to a product perspective. The goal is to merge the legacy sources into
configurable sources, keeping variant-specific functionality, to maximize reuse
and reduce maintenance effort. Thinking in terms of products rather than
projects helps to reduce the LOC and maximizes the reuse across all variants. In
order to build up software components, consisting of core assets and
variant-specific functionalities, a feature model is required for configuring
the software and thus controlling the execution flow. Although CMake is capable
of handling very simple feature models with variables and template generators
pretty well, a more complex feature model might require another tool to manage
it. Our recommendation for complex feature model configurations is KConfig, `a
domain specific language designed specifically for coding the Linux kernel
variability model'~\cite[page 3]{kconfigKernel}. KConfig is able to manage the
complex design of the Linux kernel and it has enough functionality to work with
most embedded applications. Additionally it is open-source and free to use for
anyone. Alternative solutions might also fulfill the configuration purpose.
Prominent examples are pure:variants~\cite{pureVariantsPureSystems} from pure
systems or Gears from BigLever~\cite{gearsBigLever}. Both tools are known to be
highly flexible and easy to integrate into existing
projects~\cite{confsplcGrunerBKR20}.

Independent of the feature model tool the modularization process should be the
same. All stages of the \textit{Modularization} step are visible in
Figure~\ref{fig:step2Modularization} and will be further described in detail.

\textbf{Migrate from legacy component to product component:} After the first
step \textit{Complexity Reduction}, all source code files are located in a
dedicated legacy folder. This legacy folder contains a folder for each
variant, holding all software components separately. Using a component in
multiple variants is possible, but does not make much sense. There is no reason
to share the variant component variant-a/component-1 also in variant-b. In this
step, components are just moved into a generic source folder. The folder
structure will change from:
\begin{Verbatim}[frame=single,samepage=true]
legacy/
  variant-a/
    component-1/
    component-2/
  variant-b/
    component-1/
\end{Verbatim}
to:
\begin{Verbatim}[frame=single,samepage=true]
src/
  component-1/
    variant-a/
    variant-b/
  component-2/
    variant-a/
\end{Verbatim}
By doing this, all variants of one component, with the identical or very similar
functionality, are moved to the same folder. Variants of a component residing
next to each other are easier to compare and to handle. This will enable reuse
already, but without the following stages it has no positive effect on LOC
reduction or maintenance effort.

\textbf{Remove duplicates and derive core assets:} Within the new structure
established in the previous stage, we recommend to clean up duplicate variants.
Variants of a component are considered as duplicates if their source code is
identical and no configuration is required. This is a fast and easy step for
reuse and will not require additional knowledge about a feature model or the
functionality. Removing duplicates is followed by separating core assets and
variant-specific functionality with a decorator pattern. The separation can be
done on a function level and requires static function interface definitions. The
concept is to create:
\begin{itemize}
  \item a core asset C file, which contains generic code across all variants,
        considered to be static,
  \item a header file with interface descriptions of core assets and decorators,
  \item a decorator C file for every variant, which contains a variant-specific
        implementation of a function; the function signature must be identical
        for all variants and shall be defined in the according header file.
\end{itemize}
This way the core assets and the decorators can be developed without any need of
conditional compiler directives. It is possible to implement variant-specific
differences in an object-oriented manner in C. The idea is to provide a
product-specific core asset and additionally one interface with multiple
variant-specific decorators that extend the shared functionality. The feature
model's responsibility is to take control of the underlying build system and
configure the correct decorators. If a separation of core assets and
decorators in individual files is not possible or not wanted for any reason,
another option would be to use a feature configuration with conditional
preprocessor directives or runtime configuration. Toggle points in the code make
it possible to select different implementations within the same C file.

\textbf{Refactor with focus on variability:} Last but not least, the code should
be refactored according to variability. This is the first stage that has a
functional impact on the code, and it requires a mindset change. One must stop
thinking about project-specific features and define component-specific features
relevant for the product family. Most of the legacy components do not support
variability and might require substantial refactoring to have a meaningful
feature set.

We highly encourage to implement unit tests before the refactoring starts, if
there are no unit tests available already. Refactoring can have effects on the
compiled binaries and change the behavior, even though it was not intended by
the developer. A feature model will generate many possibilities to configure the
software, therefore unit tests covering all core assets and features will ensure
that no functionality was broken and will reduce time in finding errors at a
later development phase.

\begin{figure}[htb]
  \centering
  \includegraphics[width=1.0\columnwidth]{step2-modularization-flow.pdf}
  \caption{Detailed stages of the \textit{Modularization} step}
  \Description{Detailed stages of the Modularization step.}
  \label{fig:step2Modularization}
\end{figure}

\subsection{Versioned Dependency Management}\label{dependencies}

The first two migration steps focused on the space dimension, how to organize
the source code using components and variants. This step deals with the time
dimension. By using Git it is already possible to get a variation in time. At
any created commit, it is possible to create a new branch. This branch can be
the active development branch with all the latest changes, or it can be based on
a former commit to represent a baseline of the SPL that is going to be released
or was released already. If bug fixes are required after a release, they will be
done on the same release branch. With this strategy, releases can be done on a
stable code basis without the influence of new feature developments of other
branches. Additionally the documentation of release relevant changes becomes
simple. The release branch will stay unchanged and usually not get any feature
updates, but only fixes. But release branches in a monorepo come with some
drawbacks. It is complicated to get different versions of different components.
A baseline must be done on the entire software. An integration of mixed versions
is possible but requires lots of manual effort and will likely have a strong
negative impact on maintenance. Adding a fix in a component of a specific
version in a release branch will not become available in the same version of the
same component in any other branch. All commits and branches are independent of
each other, so no automatic reuse of code between them is available. Therefore
using the same component and version provides no reuse in the space dimension if
we introduce the time dimension like this. And finally fixing a bug in a new
version, that was spotted in an earlier version, is even more complicated,
especially across all branches. Our proposal requires some more changes to the
build system and infrastructure. It is required to split up the repository
again. With the modularization being done, the split will not be done per
project anymore, but per software component. Each modularized software component
will reside in its own repository and is maintained separately. This allows us
to work with different developers or development teams on different software
components independently. To enable the development of a software component in
its own repository the following requirements must be fulfilled:
\begin{itemize}
  \item component specific requirements specifications,
  \item mocked required interfaces,
  \item unit tests for all component features in all possible configurations,
  \item software integration tests and
  \item a software test plan.
\end{itemize}
Although this step introduces a polyrepo workflow it still follows the SPL
concept by dividing products into smaller pieces still developed in SPL
repositories. The SPL repositories of the products do not contain the sources of
the isolated components anymore but each SPL repository has dependencies to
specific versions (baselines) and variants of the isolated components. These
dependencies might be references to commits of a Git repository or to released
component package artifacts, e.g., with JFrog Conan.io. This step can be done
incrementally again, component by component. The SPL repository's build system
will then do the integration job. With a proper dependency management, the build
system will integrate requested software components from external repositories
of any kind and configure them according to the feature model of the selected
variant. The benefits of this approach are the following:
\begin{itemize}
  \item Components are being developed and tested independently to get a faster
        release loop.
  \item A fix of a buggy version is implemented only once in the component's
        repository.
  \item Buggy versions can be marked in the components' repositories. With this
        variant builds containing a buggy version can print a warning or error.
        The integrator can then manually update the version of the dependency or
        justify the warning in regards to their project setup. An automatic
        update would also be possible.
\end{itemize}
Since all components are being developed independently, one package version
might not be compatible with another component's package. With Conan.io
`compatibility can even be configured and customized on a per-package
basis'~\cite{conandocs}. This is why we recommend Conan.io over Git submodules
or similar techniques. The development team together with their integrators can
configure this compatibility check. An AUTOSAR architecture might make this
check easier. An application software component should only have interfaces to
the Run-Time Environment (RTE), so based on the RTE configuration a
compatibility between components should be clearly defined.

\textbf{Cross-product component reuse:} Independent of an SPL introduction, it
is most likely that within the product portfolio of an automotive supplier like
the Marquardt GmbH with a wide range of different products, those products share
similar components and solutions. When such a company applies SPL engineering
for one of its products, sooner or later the concept is taken over for other
products of other business units, too. In case the second step of SPL migration
\textit{Modularization} is finished in at least two SPL repositories with
similar core assets and step three has been performed on a few components, the
probability, that the need to share those components between the different SPL
repositories will arise, is rather high. Therefore the third step of our
migration concept provides an additional useful feature: isolation of
configurable sources for \textit{Cross-product component reuse}. The idea is
that the isolated software component and all its variants can be used in
multiple other SPL repositories, not just in one. There are functionalities,
e.g., current or voltage measurement, that might be useful in a multitude of
electronic products. This can exponentially increase reuse in the entire product
portfolio.

\section{Conclusion and Future Work}\label{conclusion}

This work provides an iterative and incremental migration strategy towards SPL
for the software of a real-world product family of an automotive supplier. We
proposed three steps to perform the migration of a project's software repository
to an SPL variant: (1) \textit{Complexity Reduction}, (2)
\textit{Modularization} and (3) \textit{Versioned Dependency Management}. It was
shown that our proposal can be applied to the repositories of a product family's
software at any time of the development phase due to the fact that any of the
three steps can be applied incrementally and each transformed variant can
iteratively continue with the next migration step independent of the others.
With our proposal we:

\begin{itemize}
  \item established a stable SPL with variants buildable at any time,
  \item reduced workload of developers by reducing the number of repositories,
  \item increased collaboration beyond project borders and
  \item integrated all software sources of a product family into one git
        repository.
\end{itemize}

We implemented a build system with modern CMake fully supporting the concept of
SPL variants.

At the submission date's point of time we managed to proof about 50\% of our
concept. The build system prototype required two persons full-time for six
weeks, the implementation for migration step one and two required two persons
full-time for three more months. Over a time frame of two months we
incrementally added ten variants, while each import to the SPL was really fast
due to the automated importing mechanism. We already started with
\textit{Modularization} of software components, but only converted a few
components and begun with removing duplicates. Recently we focussed more on
automation than on the SPL migration, as we saw more advantages there in the
long run. The automation for CI/CD is helpful for the migration as well.
Nevertheless, the build system requires some more features to be able to handle
our migration step three and we plan to implement them in the coming months, in
parallel to the \textit{Modularization} step of the software components.

Additionally to the build system implementation and the actual migration of the
embedded source code, it is required for us to qualify tools which have an
impact on the created binaries or the processes to build them. This tool
qualification is requested by the ISO 26262 standard for safety reasons. We must
take care that CMake, Ninja and KConfig are properly qualified. On the other
hand, proprietary software might come with safety evaluations already. This
should be kept in mind, when deciding for your tool landscape.

Besides all the discussed technical challenges, there were also social obstacles
on the way to SPL. Fear was probably the strongest emotion we faced. At the
beginning we were certain that all required build system features were
implemented and we always had a backup plan to go back to the original
repositories. There was no real threat and still no one had enough confidence to
start the migration. Afterwards we do not see a reason for hesitation, but
sometimes it simply needs someone brave in the team or someone pushing you. Our
conclusion is: Just do it! And if you break it, make sure you can fix it fast!


\balance{}

\bibliographystyle{ACM-Reference-Format}

%%%%%%%%%%%%%%%%%%%%%%%%%%%%%%%%%%%%%%%%%%%%%%%%%%%%%%%%%%%%%%%%%
%%% place references in references.bib for better structuring %%%
%%%%%%%%%%%%%%%%%%%%%%%%%%%%%%%%%%%%%%%%%%%%%%%%%%%%%%%%%%%%%%%%%
\bibliography{references}

\end{document}