\section{Challenge}

Our migration method is supposed to be a generic solution, but our company's
circumstances might have some influence on the ideas that we share in this
paper. This section will explain the challenges we had, both coming from the
products' nature and its industry and also from our corporate processes and
strategies.

The projects that we migrated are in the area of electric cars for premium
automobile manufacturers. The software is written in (Embedded) C and Matlab and
modelled with AUTomotive Open System ARchitecture (AUTOSAR) with C code
generators. These projects must fulfill Automotive Software Process Improvement
and Capability Determination (ASPICE), ISO26262 and other processes, standards
and regulations of the automotive industry during their development phases to
stay competitive and meet legal requirements. A company will less likely get a
new inquiry when the standards and norms are not fulfilled. The standards are
relevant for the entire SPL development and do not only apply for the delivered
software product, but also for supporting processes and tools like the build
system. Additionally, the build system must be able to handle third party source
code deliveries without modifications. One example for this are AUTOSAR
deliveries. The dependency to a delivered AUTOSAR package is not only a
challenge for the toolchain, but also a chance. Due to AUTOSAR's layered
architecture and its interface and component design, it already provides modular
abstraction for separation of concerns, that can help in building SPL core
assets and to maximize functional reuse.

All migrated projects of our SPL base on the same product, a complex sensor
cluster. For historic reasons every (customer) project was managed as a separate
Dimensions repository. Dimensions is a Source Code Management (SCM) system
similar to Revision Control System (RCS) used at the Marquardt GmbH for source
code administration and project documentation. Each individual source code
repository is a self-contained ready to build project, including all required
tools and libraries to build the product binaries, like compilers, MSYS and GNU
Make. The C code was validated with Jenkins nightly builds. Those Jenkins
scripts were located in separate repositories, not part of the project's
repository itself. Unit tests were not running with every change of the source
code as part of CI/CD, but were running on demand.

Whenever a new customer project was kicked-off, it also triggered the start of a
new software project. This software project then started as a clone of another
already existing project (`clone-and-own'). This approach had several benefits:
the setup time was short and because a mature project was taken as basis, the
project's software was stable already and debugging and adapting was possible.
There was less work for generic parts and only hardware and customer specific
modifications were required. However, this approach also caused a lot of
disadvantages. By simply cloning an existing project there was not only a copy
of functionality available in the new project, but also a copy of all its flaws.
Because of a large number of copies a possible refactoring of reused code had to
be merged back and this meant a high effort. If refactoring is never done, this
approach can result in software erosion, as it did in our projects. The software
erosion noticeably led to higher maintenance efforts at the end phase of the
development and brought up errors late during the product lifecycle. For all the
teams working in these projects, with the problems mentioned above, the idea of
migrating to another platform including a new tool chain, not only sounded like
a mammoth task but unrealistic too, having in mind the tightly planned release
schedules.

With our migration to SPL we try to tackle as many as possible of the mentioned
challenges. The migration is only partially done and still ongoing, but we did
not detect major issues so far. At the same time, we managed to:
\begin{itemize}
  \item migrate `clone-and-own' sources to our SPL,
  \item add variant handling capability to the build system,
  \item replace the outdated SCM system by Git and Bitbucket,
  \item introduce a CI/CD system,
  \item introduce a unit test framework as integral part of the development,
  \item automate the setup of the build environment and reduce the repository size drastically,
  \item stay compliant with all rules and norms,
  \item train the team and
  \item still keep release plans.
\end{itemize}

We will not write about all solution aspects in this paper. The focus will be
on the source code migration process and less on the tools, processes and
performed trainings. Our build system implementation is freely available on
Github.com~\cite{GithubSPL}.
